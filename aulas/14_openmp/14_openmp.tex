
\documentclass[xcolor={usenames,dvipsnames},12pt,presentation,aspectratio=169]{beamer}

\usepackage[utf8]{inputenc}
\usepackage{fontawesome}
\usepackage[brazilian]{babel}
\usepackage{verbatim}
\usepackage{graphicx}
\usepackage{xspace}
\usepackage{amsthm}
\usepackage{url}
\usepackage{array}
\usepackage{hyperref}
\usepackage{times,mathptmx}
\usepackage{pdfpages}
\usepackage{mdframed}
\usepackage{tikz}
\usepackage{alltt}
\usepackage{minted}
%\usepackage{times}
%\usepackage[usenames,dvipsnames]{xcolor}
%\usepackage[usenames,dvipsnames]{color}
%\usepackage{color}

\usetikzlibrary{arrows,shapes}

\usetheme{Madrid}
%\usetheme{Boadilla}
%\usetheme{Darmstadt}
%\usetheme{Frankfurt}
%\usetheme{CambridgeUS}
%\usetheme{AnnArbor}
%\usecolortheme{beaver}
%\usecolortheme{seahorse}
%\usecolortheme{seagull}
\usecolortheme[named=BrickRed]{structure}

\setbeamercovered{transparent}

\setbeamertemplate{footline}[frame number]
%\setbeamertemplate{navigation symbols}{}
%\setbeamersize{text margin left=1em,text margin right=1em}

\newcommand{\titulo}{Programação Paralela com OpenMP}
\newcommand{\disciplina}{ELC139 - Programação Paralela}
\newcommand{\nome}{João Vicente Ferreira Lima (UFSM)}

\lecture[1]{\aula}{aula01}
\def\lecturename{\aula}

\newcommand{\Red}[1]{{\color{red}#1}}
\newcommand{\red}[1]{{\color{red}#1}}
\newcommand{\Blue}[1]{{\color{blue}#1}}
\newcommand{\blue}[1]{{\color{blue}#1}}

\newcommand{\PBS}[1]{\let\temp=\\#1\let\\=\temp}
\newcommand{\RRCOL}{\PBS\raggedright\hspace{0pt}}

\newcommand{\p}[1]{\texttt{#1}}
\newenvironment{code}{%
  \begin{alltt}%
  }{%
  \end{alltt}%
}

% https://github.com/joao-lima/ATPESC

\makeatletter
%\setbeamertemplate{headline}{}
% {%
%   \leavevmode%
%   \@tempdimb=2.4375ex%
%   \ifnum\beamer@subsectionmax<\beamer@sectionmax%
%     \multiply\@tempdimb by 4%
%   \else%
%     \multiply\@tempdimb by\beamer@subsectionmax%
%   \fi%
%   \ifdim\@tempdimb>0pt%
%     \advance\@tempdimb by 1.125ex%
%     \begin{beamercolorbox}[wd=.5\paperwidth,ht=\@tempdimb]{section in head/foot}%
%       \vbox to\@tempdimb{\vfil\insertsectionnavigation{.5\paperwidth}\vfil}%
%     \end{beamercolorbox}%
%     \begin{beamercolorbox}[wd=.45\paperwidth,ht=\@tempdimb]{subsection in head/foot}%
%       \vbox
%       to\@tempdimb{\vfil\insertsubsectionnavigation{.45\paperwidth}\vfil}%
%     \end{beamercolorbox}%
%     \begin{beamercolorbox}[wd=.05\paperwidth,ht=\@tempdimb]{subsection in head/foot}%
%       \vbox
%       to\@tempdimb{\vfil\hfil\insertframenumber\vfil\vfil}%
%     \end{beamercolorbox}%
%   \fi%
% }

\def\dohead{\beamer@headcounter=4\relax\beamer@headcounter=1\loop\ifnum\beamer@headcounter<\beamer@totalheads%
  \advance\beamer@headcounter by1\relax%
  \csname @@head\the\beamer@headcounter\endcsname\repeat}

\makeatother

\title[\titulo]{\titulo}

\subtitle{\disciplina}

\author[João V. F. Lima]{\nome}

%\institute[UFSM]{Departamento de Linguagens e Sistemas de Computação \\ Universidade Federal de Santa Maria \\ \url{jvlima@inf.ufsm.br} \\ \url{http://www.inf.ufsm.br/~jvlima}}
\institute[UFSM]{Universidade Federal de Santa Maria \\ \url{jvlima@inf.ufsm.br} \\ \url{http://www.inf.ufsm.br/~jvlima}}
\date{2023/1}

\graphicspath{{.}{figs/}}

\logo{ \includegraphics[height=1.5cm,width=1.5cm,keepaspectratio]{logo_inf}    
        \includegraphics[height=1.5cm,width=1.5cm,keepaspectratio]{logo_ufsm} }

%\titlegraphic{
%	\includegraphics[width=2cm]{logo_ufsm}
%  \hspace{1cm}
%	\includegraphics[width=2cm]{logo_inf}
%}

\newtheorem{mydef}{Definição}[section]
%\newtheorem{myteo}{Teorema}[section]
%------------------------------------------------------------------------------
%\newcommand{\xkaapi}{XKaapi\xspace}
%------------------------------------------------------------------------------
% Typesetting Listings
\usepackage{listings}
\lstset{
  language=C,
  %basicstyle=\scriptsize\ttfamily,
  %basicstyle=\normalsize\ttfamily,
  basicstyle=\small\ttfamily,
  %basicstyle=\footnotesize\ttfamily,
  %aboveskip=0pt,
  %belowskip=0pt,
  %mathescape=false,
  columns=fullflexible,
  %numbers=none,
  numbers=left,
  numbersep=5pt,
%  showtabs=true,
%  showspaces=true,
  frame=tb,
  breaklines=true
}
%------------------------------------------------------------------------------
%\lstset{commentstyle=\color{blue}}
%\lstset{stringstyle=\ttfamily}
%\lstset{ classoffset=1, 
%            morekeywords={kaapi,omp,task,data,alloca, declare, reduction, identity, parallel,sync,taskwait,cilk,spawn,tbb,css,cilk\_spawn,cilk\_sync,cilk\_for,offload},
%            keywordstyle=\color{Red}\bfseries
%           }
%\lstset{ classoffset=2, 
%            morekeywords={value,read,write,readwrite,reduction,untied,firstprivate,TaskBodyCPU,TaskBodyGPU,ka,Signature,RW,CW,range2d\_r,range2d\_rw,range2d,Spawn,Fork,Shared\_w,Shared\_r,Shared,a1,target,device,copyin,copyout,input,implements,copy\_deps,RPWP,range2d\_rpwp,rangeindex,Memory,Register,SetStaticSched,Sync,Unregister,Community,System,join\_community,SpawnMain,leave,initialize,terminate,logfile,array,SetArch,ArchHost,ArchCUDA,W,R,gpuStream,pointer\_w,pointer\_r,pointer\_cw,pointer},
%            keywordstyle=\color{Blue}\bfseries
%           }
%\lstset{ classoffset=3, 
%            morekeywords={storage,ld},
%            keywordstyle=\bfseries
%           }
%\lstset{ classoffset=4, 
%            morekeywords={in,out,inout,cout,concurrent},
%            keywordstyle=\color{Red}\bfseries
%           }
%           
%\lstset{classoffset=0, showstringspaces=false}
%------------------------------------------------------------------------------
\mdfsetup{
  backgroundcolor=gray!10,
%  roundcorner=10pt,
}
%------------------------------------------------------------------------------
\newcommand{\restorefootline}{\setbeamertemplate{navigation symbols}{}}
%\newcommand{\setfootline}[1]{\setbeamertemplate{navigation symbols}{\textcolor{black}{\textbf{#1}}}}
\newcommand{\includeslides}[4]{%
%  \setfootline{#1}%
  {
    \setbeamercolor{background canvas}{bg=}
    \includepdf[pages={#1},%
    pagecommand={},
%    pagecommand={\begin{frame}[default]{}\end{frame}},
%    #4,%
    turn=false,noautoscale=false,column=false,columnstrict=false,openright=false,frame=false]{#2}%
  }
  %\restorefootline%
}
%------------------------------------------------------------------------------
\begin{document}

\begin{frame}
%  \titlepage
  \maketitle
%  \mode<presentation>
%  {
%    \begin{columns}
%      \begin{column}{0.5\textwidth}
%      \raggedleft
%	\includegraphics[width=2cm]{logo_ufsm}
%      \end{column}
%      \begin{column}{0.5\textwidth}
%	\includegraphics[width=2cm]{logo_inf}
%      \end{column}
%    \end{columns}
%  }
\end{frame}

\begin{frame}
    \frametitle{Outline}
%    \tableofcontents[hideallsubsections]
    \tableofcontents
\end{frame}

\AtBeginSection{
  \begin{frame}
    \frametitle{Outline}
    \tableofcontents[currentsection]
  \end{frame}
}

\AtBeginSubsection[]
{
    \begin{frame}
        \frametitle{Outline}
        \tableofcontents[currentsection,currentsubsection]
    \end{frame}
}

%\begin{minted}[linenos, fontsize=\small, breaklines=true, frame=lines]{C}
%\end{minted}

%%%%%%%%%%%%%%%%%%%%%%%%%%%%%%%%%%%%%%%%%%%%%%%%%%%%%%%%%%%%%%%%%%%%%%%%%%%%%%%
\section{Introdução}
%%%%%%%%%%%%%%%%%%%%%%%%%%%%%%%%%%%%%%%%%%%%%%%%%%%%%%%%%%%%%%%%%%%%%%%%%%%%%%%
\begin{frame}[fragile]
  \frametitle{Exemplo: Pi}
\begin{minipage}{0.95\textwidth}  
\begin{minted}[linenos, fontsize=\small, breaklines=true, frame=lines]{C} 
for (i=1;i<= num_steps; i++){
  x = (i-0.5)*step;
  sum = sum + 4.0/(1.0+x*x);
}
\end{minted}
\end{minipage}
%
\end{frame}
%%%%%%%%%%%%%%%%%%%%%%%%%%%%%%%%%%%%%%%%%%%%%%%%%%%%%%%%%%%%%%%%%%%%%%%%%%%%%%%
\begin{frame}[fragile]
  \frametitle{Exemplo: Pi OpenMP}
\begin{itemize}
  \item \mintinline{C}|x| precisa ser privado
  \item \mintinline{C}|sum| é somado por todos
\end{itemize}
\begin{minipage}{0.95\textwidth}  
\begin{minted}[linenos, fontsize=\small, breaklines=true, frame=lines]{C} 
#pragma omp parallel for private(x) reduction(+:sum)
for (i=1;i<= num_steps; i++){
  x = (i-0.5)*step;
  sum = sum + 4.0/(1.0+x*x);
}
\end{minted}
\end{minipage}
%
\end{frame}
%%%%%%%%%%%%%%%%%%%%%%%%%%%%%%%%%%%%%%%%%%%%%%%%%%%%%%%%%%%%%%%%%%%%%%%%%%%%%%%
\begin{frame}[fragile]
  \frametitle{Exemplo: Histograma}
\begin{minipage}{0.95\textwidth}  
\begin{minted}[linenos, fontsize=\small, breaklines=true, frame=lines]{C} 
for(int i=0;i<num_trials;i++){
    
  long ival = (long) (x[i] - xlow)/bucket_width;

  hist[ival]++;  
}
\end{minted}
\end{minipage}
%
\end{frame}
%%%%%%%%%%%%%%%%%%%%%%%%%%%%%%%%%%%%%%%%%%%%%%%%%%%%%%%%%%%%%%%%%%%%%%%%%%%%%%%
\begin{frame}[fragile]
  \frametitle{Exemplo: Histograma OpenMP}
\begin{minipage}{0.95\textwidth}  
\begin{minted}[linenos, fontsize=\small, breaklines=true, frame=lines]{C} 
#pragma omp parallel for
for(int i=0;i<num_trials;i++){
    
    long ival = (long) (x[i] - xlow)/bucket_width;

    #pragma omp critical
        hist[ival]++;   
}
\end{minted}
\end{minipage}
%
\end{frame}
%%%%%%%%%%%%%%%%%%%%%%%%%%%%%%%%%%%%%%%%%%%%%%%%%%%%%%%%%%%%%%%%%%%%%%%%%%%%%%%
\begin{frame}[fragile]
  \frametitle{Exemplo: Histograma OpenMP}
\begin{minipage}{0.95\textwidth}  
\begin{minted}[linenos, fontsize=\small, breaklines=true, frame=lines]{C} 
#pragma omp parallel for reduction(+:hist[0:num_buckets])
for(int i=0;i<num_trials;i++){
    
    long ival = (long) (x[i] - xlow)/bucket_width;

    hist[ival]++; 
}
\end{minted}
\end{minipage}
%
\end{frame}
%%%%%%%%%%%%%%%%%%%%%%%%%%%%%%%%%%%%%%%%%%%%%%%%%%%%%%%%%%%%%%%%%%%%%%%%%%%%%%%
\begin{frame}[fragile]
  \frametitle{Exemplo: Jacobi}
  \begin{itemize}
    \item $Ax = b$
  \end{itemize}
%
\begin{columns}
  \begin{column}{0.6\textwidth} 
\begin{minipage}{0.95\textwidth}  
  \begin{minted}[linenos, fontsize=\scriptsize, breaklines=true, frame=lines]{C} 
while((conv > TOLERANCE) && (iters<MAX_ITERS))
{
  iters++;
  xtmp  = xnew;   // don't copy arrays.
  xnew  = xold;   // just swap pointers.
  xold  = xtmp;

  for (i=0; i<Ndim; i++){
      xnew[i] = (TYPE) 0.0;
      for (j=0; j<Ndim;j++){
          if(i!=j)
            xnew[i]+= A[i*Ndim + j]*xold[j];
      }
      xnew[i] = (b[i]-xnew[i])/A[i*Ndim+i];

  }
  \end{minted}
  \end{minipage}    
  \end{column}
  \begin{column}{0.4\textwidth}
\begin{minipage}{0.95\textwidth}  
  \begin{minted}[linenos, fontsize=\scriptsize, breaklines=true, frame=lines]{C} 
  //
  // test convergence    
  //
  conv = 0.0;
  for (i=0; i<Ndim; i++){
      tmp  = xnew[i]-xold[i];
      conv += tmp*tmp;
  }
  conv = sqrt((double)conv);
}
  \end{minted}
  \end{minipage}      
  \end{column}
\end{columns}
%
\end{frame}
%%%%%%%%%%%%%%%%%%%%%%%%%%%%%%%%%%%%%%%%%%%%%%%%%%%%%%%%%%%%%%%%%%%%%%%%%%%%%%%
\begin{frame}[fragile]
  \frametitle{Exemplo: Jacobi OpenMP}
%
\begin{columns}
  \begin{column}{0.6\textwidth} 
\begin{minipage}{0.95\textwidth}  
\begin{minted}[linenos, fontsize=\scriptsize, breaklines=true, frame=lines]{C} 
while((conv > TOLERANCE*TOLERANCE) && (iters<MAX_ITERS))
{
  {
      iters++;
      conv = 0.0;
      xtmp  = xnew;   // don't copy arrays.
      xnew  = xold;   // just swap pointers.
      xold  = xtmp;
  }

  #pragma omp parallel for private(i,j)
  for (i=0; i<Ndim; i++){
      xnew[i] = (TYPE) 0.0;
      for (j=0; j<Ndim;j++){
        xnew[i]+= A[i*Ndim + j]*xold[j] * (i != j);
      }
      xnew[i] = (b[i]-xnew[i])/A[i*Ndim+i];

  }
\end{minted}
  \end{minipage}    
  \end{column}
  \begin{column}{0.4\textwidth}
\begin{minipage}{0.95\textwidth}  
\begin{minted}[linenos, fontsize=\scriptsize, breaklines=true, frame=lines]{C} 
  //  
  // test convergence
  //
  #pragma omp parallel for private(tmp) reduction(+:conv)
  for (i=0; i<Ndim; i++){
      tmp  = xnew[i]-xold[i];
      conv += tmp*tmp;
  }
}
\end{minted}
  \end{minipage}      
  \end{column}
\end{columns}
%
\end{frame}
%%%%%%%%%%%%%%%%%%%%%%%%%%%%%%%%%%%%%%%%%%%%%%%%%%%%%%%%%%%%%%%%%%%%%%%%%%%%%%%
\begin{frame}[fragile]
  \frametitle{Exemplo: Jacobi OpenMP}
%
\begin{columns}
  \begin{column}{0.6\textwidth} 
\begin{minipage}{0.95\textwidth}  
\begin{minted}[linenos, fontsize=\scriptsize, breaklines=true, frame=lines]{C} 
#pragma omp parallel default(none) private(tmp) \
  shared (Ndim, conv, iters, b, A, xnew, xold, xtmp) 
{
  while((conv > TOLERANCE*TOLERANCE) && (iters<MAX_ITERS))
  {
    #pragma omp single
    {
      xtmp  = xnew;   // don't copy arrays.
      xnew  = xold;   // just swap pointers.
      xold  = xtmp;
    }
    #pragma omp for private(i,j) nowait
    for (i=0; i<Ndim; i++){
      xnew[i] = (TYPE) 0.0;
      for (j=0; j<Ndim;j++){
        xnew[i]+= A[i*Ndim + j]*xold[j] * (i != j);
      }
      xnew[i] = (b[i]-xnew[i])/A[i*Ndim+i];
    }
\end{minted}
  \end{minipage}    
  \end{column}
  \begin{column}{0.4\textwidth}
\begin{minipage}{0.95\textwidth}  
\begin{minted}[linenos, fontsize=\scriptsize, breaklines=true, frame=lines]{C} 
  #pragma omp single
  {
    iters++;
    conv = 0.0;
  }

  // test convergence
  #pragma omp for private(tmp) reduction(+:conv)
  for (i=0; i<Ndim; i++){
    tmp  = xnew[i]-xold[i];
    conv += tmp*tmp;
  }
}
}
\end{minted}
  \end{minipage}      
  \end{column}
\end{columns}
%
\end{frame}
%%%%%%%%%%%%%%%%%%%%%%%%%%%%%%%%%%%%%%%%%%%%%%%%%%%%%%%%%%%%%%%%%%%%%%%%%%%%%%%
%------------------------------------------------------------------------------
\begin{frame}[plain]{}
  \begin{center}
    \vspace{2cm}
    \Large{https://joao-ufsm.github.io/par2023a/}
    
    \vspace{1cm}
    \includegraphics[width=2cm]{logo_ufsm}
    \hspace{0.5cm}
    \includegraphics[width=2cm]{logo_inf}
  \end{center}
\end{frame}
%------------------------------------------------------------------------------

\end{document}